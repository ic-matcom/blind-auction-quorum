\begin{resumen}
	Las subastas son un mecanismo de venta cada día más usado. Con el auge de las tecnologías de la información, sistemas de pago electrónico y de un mundo interconectado, a través de internet, cada vez se hace un mayor uso de subastas electrónicas. Por esta razón, se hace necesario proponer nuevos protocolos y mecanismos para incrementar la seguridad de estos sistemas y protegerlos de hackers que explotan vulnerabilidades de seguridad en las plataformas en línea, pero también proteger a los usuarios de subastadores malintencionados que pueden terminar siendo estafadores, este último problema cada vez más latente en la actualidad. En la presente investigación se hace un estudio de los protocolos de una subasta en particular, las subastas a ciegas sobre blockchain. Luego de revisar y valorar las propuestas de la literatura consultada sobre el tema, se hace una propuesta e implementación de un contrato inteligente que permite la realización de una variante de subasta holandesa (a ciegas), en el lenguaje de programación Solidity, que pudiera ser utilizada en el Mercado de Deuda Pública de Cuba. Se valora, además, la factibilidad de la blockchain de Quorum como blockchain objetivo para desplegar la propuesta presentada. Y por último, se realiza un despliegue del contrato inteligente en una red blockchain local, en la cual se realizaron pruebas que fueron satisfactorias, llegando a la conclusión de que la propuesta es factible y puede ser empleada para realizar subastas a ciegas.

\end{resumen}

\begin{abstract}
	Auctions are an increasingly used sales mechanism. With the rise of information technologies, electronic payment systems and an interconnected world, through the Internet, there is increasing use of electronic auctions. For this reason, it is necessary to propose new protocols and mechanisms to increase the security of these systems and protect them from hackers who exploit security vulnerabilities in online platforms, but also protect users from malicious auctioneers who may end up being scammers. This last problem is increasingly common today. In the present investigation, a study is made of the protocols of a particular auction, blind auctions on blockchain. After reviewing and evaluating the proposals of the literature consulted on this topic, a proposal and implementation of a smart contract is made that allows the realization of a variant of the Dutch (blind) auction, in the Solidity programming language, which could be used in the Cuban Public Debt Market. In addition, the feasibility of the Quorum blockchain as the blockchain to deploy the presented proposal is assessed. And finally, a deployment of the smart contract is carried out in a local blockchain network, in which tests were carried out that were satisfactory, reaching the conclusion that the proposal is feasible and can be used to carry out blind auctions.
\end{abstract}
