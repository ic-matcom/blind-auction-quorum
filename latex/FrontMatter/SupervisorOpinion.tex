\begin{opinion}
    El tema que ocupa al estudiante, las subastas a ciegas en entornos digitales, ha
    cobrado interés tras la aparición de las tecnologías basadas en Blockchain. Como
    bien se demuestra en el trabajo de diploma, por la cantidad de publicaciones
    citadas al respecto, es un tema actual de interés. Existen un número importante
    de compañías trabajando en lograr sistemas para subastas electrónicas justos y
    confiables, potenciados por las grandes casas de subastas.

    En el caso particular del Instituto de criptografía, el interés radica en que se está
    desarrollando una plataforma digital para el mercado de deuda pública, que requiere
    de la implementación de subastas. La implementación de sistemas de este tipo, trae
    consigo un conjunto de problemas que requieren soluciones novedosas, como en este
    caso.
    
    En la concepción del trabajo el estudiante puso en práctica los conocimientos
    adquiridos a lo largo de la licenciatura y adquirió otros como los referentes a la
    criptografía de clave pública, asimiló el lenguaje de programación Solidity. Mostró
    independencia durante su desempeño, poder de síntesis y capacidad de análisis. Fue
    receptivo a las indicaciones y señalamientos realizados por su dirigente.
    
    El contenido del trabajo desarrollado por el estudiante está en total
    correspondencia con la tarea planteada. Con la solución propuesta se cumplen los
    objetivos propuestos.

    El documento presentado cumple con las normas de redacción que se exigen, se
    estructura correctamente, las ideas se exponen con claridad, haciendo citas de manera
    oportuna, empleando gráficos cuando se requiere.
    
    Otro aspecto a destacar es que se llegó a la implementación, se realizaron
    pruebas unitarias a los algoritmos programados y se verificó su posible empleo en la
    plataforma digital para el mercado de deuda pública que se desarrolla en el instituto.
    Por todo lo planteado se propone al educando la calificación \textbf{ “Sobresaliente” (5 puntos)}

    \begin{flushright}
        \textbf{Dr.C. Yaidir Mustelier Ruiz}
    \end{flushright}

\end{opinion}