\chapter{Propuesta}\label{chapter:proposal}

  En este capítulo se describirá el algoritmo implementado para ejecutar una variante de subasta holandesa
a ciegas sobre blockchain, en este caso para una red blockchain basada en Ethereum, programada en el 
lenguaje de programación solidity. Este algoritmo se hace con la finalidad de facilitar y realizar de manera
segura la venta de bonos soberanos para el mercado de deuda pública. 

  \section{Ventajas de la propuesta implementada}
    Dado que los esquemas y protocolos para subastas a ciegas de la literatura consultada no cumplen con los requerimientos necesarios
    para resolver el problema en cuestión (subasta a ciegas de bonos soberanos) o se hace difícil su adaptación al problema a resolver.
    Se propone una solución que se adapta a las necesidades del problema y que además es escalable, segura y confiable.

    Ventajas

    \begin{enumerate}
      \item No es necesario que los participantes de la subasta se registren en esta, para participar solamente necesitan hacer dos 
      transacciones, la oferta y luego la revelación de la oferta.
      \item Posibilidad de hacer varias ofertas. Cada participante o postor de la subasta puede realizar cuantas ofertas estime convenientes,
      no tiene limitantes en cuanto al número de ofertas.
      \item Permite retractar o cancelar ofertas. En la fase de revelación el usuario puede decidir no proceder con una oferta.
      \item Admite múltiples ganadores. Los bonos se reparten entre los ganadores de la subasta, que en la mayoría de los casos será más 
      de uno. 
    \end{enumerate}

    Desventajas:
  
    \begin{enumerate}
      \item Las ofertas son reveladas. Las ofertas, si bien en la fase de ofertas son desconocidas, es necesario revelar la información
      real de la oferta para comprobar su validez.
      \item Posibilidad de colisión. Al utilizar un algoritmo de \textit{hash} para codificar las ofertas, existe la posibilidad (aunque bastante
      poco probable) de que la información de dos ofertas diferentes, den el mismo \textit{hash}. 
    \end{enumerate}

    Teniendo en consideración las ventajas y desventajas anteriormente mencionadas, el autor cree que las desventajas no representan un
    problema tan significativo, dado que la seguridad y confiabilidad de la subasta se mantienen en un nivel alto. Y, por el contrario, tiene
    ventajas muy beneficiosas para resolver la problemática, incluso alguna de ellas no vistas en los esquemas de subasta que se han 
    investigado con anterioridad.

    \section{Condiciones iniciales}
    Para mejor entendimiento del lector y lograr ser más específicos en algunos asuntos, la propuesta estará enfocada a la blockchain de
    Ethereum, y todo lo que se refiera a la implementación e interacción del contrato inteligente a partir de ahora, va a estar enfocada
    a esta blockchain en particular. A pesar de esto, la propuesta implementada se podrá utilizar en la red de Quorum o en otras redes basadas
    en Ethereum sin ninguna o muy pocas modificaciones.

    Para participar en la subasta como postor solo es necesario tener una cuenta en la blockchain de Ethereum, la aplicación o extensión
    Metamask (o alguna otra que permita interactuar con contratos inteligentes) y una cantidad de ether suficiente para pagar la comisión
    de gas de las transacciones.

    \section{Proceso de Subasta}
    El proceso de la subasta va a estar compuesto por 5 fases principales: despliegue, ofertas, revelación,
    verificación y finalización.

    Para mejor entendimiento, a partir de este momento al que oferta los bienes a subastar (en este caso los bonos) se le llamara 
    beneficiario (para el problema en cuestión sería el gobierno), al que despliega el contrato inteligente en la blockchain le llamaremos 
    subastador (a pesar de que no cumple el mismo objetivo ni funciona como los subastadores tradicionales). El beneficiario y el 
    subastador pueden ser el mismo o personas diferentes. Por último, a los participantes en la subasta, se les llamará postores. 
    Beneficiario, subastador y postores van a ser los usuarios de la subasta para mayor comodidad.

    \subsection{Desplegar contrato}
      El contrato inteligente puede ser desplegado a la blockchain por el propietario de lo que se oferta
      o por una tercera persona que haga función de subastador. Esta persona no tiene ningún poder, ni ningún
      privilegio en el contrato inteligente, ni tampoco tiene acceso a retirar los activos del contrato. La única función del subastador 
      es la de poner los parámetros iniciales de esta, dígase: 
    
      \begin{enumerate}
        \item boneToSale: valor total de los bonos que se quieren vender
        \item biddingTime: tiempo de duración de la fase de ofertas
        \item revealTime: tiempo de duración de la fase de revelación de ofertas
        \item beneficiaryAddress: dirección donde se quiere recibir el pago de los bonos vendidos.
      \end{enumerate}

      Cada vez que se quiere hacer una nueva venta de bonos, es necesario volver a desplegar el contrato inteligente.

    \subsection{Fase de ofertas}
      Cada postor, puede realizar cuantas ofertas estime convenientes. Dado que lo que se oferta en la subasta son 
      bonos, que como se explicó anteriormente, son una especie de préstamos por tiempo definido que se le hace al gobierno; cada puja está
      compuesta de dos partes, la cantidad que el postor está dispuesto a prestar y cual sería el interés a cobrar por ese préstamo en 
      porciento. 

      Dada la necesidad de que las ofertas no sean conocidas por los demás postores, luego de escoger las condiciones de la oferta a 
      efectuar, 
      en vez de enviar los datos reales de esta, el postor codifica esa información (cantidad y porciento) en conjunto con una clave
      privada solo conocida por él, y está codificación es lo que se envía al contrato inteligente. Es importante destacar que la cantidad
      enviada en la transacción, en conjunto con la codificación de la oferta, no tiene que ser necesariamente la cantidad exacta de la 
      oferta enviada, esta cantidad se deposita en el contrato inteligente y se añade al saldo de esa dirección en el contrato, para un 
      posible uso posterior de este en ofertas venideras.

      Para codificar la oferta se hace uso de la función de solidity: \textit{keccak256}, la cual es un algoritmo o función de 
      \textit{hash} que toma como entrada un conjunto de datos y devuelve un valor de longitud fija, 32 bytes. 
      Esta función es una de las más utilizadas en la programación de contratos inteligentes, ya que permite la creación de 
      \textit{hash} de datos que no pueden ser revertidos a su valor original, es decir, que no se puede obtener la información original
      de un \textit{hash} de esta función. Esta función es empleada para codificar la oferta, por el hecho de que al ser una función de hash,
      no se puede obtener la información original de la oferta, por lo que no se puede saber la cantidad y el porciento de la oferta
      que se realizó.

      Luego de desplegado el contrato inteligente hay un tiempo hábil para mandar ofertas, luego de ese tiempo no serán recibidas más 
      ofertas. Y da inicio a la fase de revelación de ofertas.

      \subsection{Revelación de ofertas}
      Cada postor que realizó ofertas debe enviar al contrato la información de las ofertas que quiere revelar. Es decir, tiene que enviar
      tres arreglos, que van a representar valor, porciento y clave secreta de las ofertas, respectivamente. Es necesario que los tamaños
      de los arreglos sean igual a la cantidad de ofertas enviadas, de lo contrario, no serán analizados, y será necesario una nueva 
      transacción con la información completa. La posición del arreglo 
      significa el número de la oferta, ordenada por tiempo de recepción de la oferta por el contrato inteligente. El contrato inteligente 
      se encarga de codificar la información suministrada, con la misma función \textit{keccak256} que fue utilizada por el postor y la 
      comprueba con la codificación enviada por este en la fase de ofertas. Si las dos codificaciones coinciden exactamente, quiere decir, 
      que los datos de la oferta son los mismos que el postor eligió en la fase de ofertas y la oferta se considera válida. Para que la
      oferta sea totalmente válida, es necesario que el valor depositado hasta ese momento en el contrato sea mayor que el valor de la
      puja de esa transacción. A cada oferta válida se le asigna un identificador único (id), por orden de revelación (las ofertas que 
      primero se revelan tienen un menor id), que posteriormente será usado.

      Si la oferta no es válida, ya sea por no coincidir los valores \textit{hash} de las codificaciones o por no tener suficiente 
      dinero disponible para ejecutar la oferta, se anula la oferta, sin embargo, el dinero depositado en esa transacción queda disponible
      para próximas ofertas, aunque también disponible para retirar en cualquier momento posterior. Cuando se solicita un retiro 
      (\tetxit{withdraw}) del contrato, si la dirección que lo solicita tiene algún fondo disponible para retiro, pues recibe el reembolso
      de todo lo disponible en el contrato.

      Es necesario destacar que la revelación de las ofertas ocurre solamente una única vez por cada dirección, es decir, si una oferta
      es considerada válida o no válida (por alguna de las razones vistas anteriormente) pues no hay forma de cambiar ese veredicto. Por 
      esto es necesario tener sumo cuidado con la información que se envía al contrato inteligente tanto en la fase de ofertas, como en
      la fase de revelación, ya que una vez que se envía, no hay forma de cambiarla.

    \subsection{Verificación y publicación de los ganadores}
      En esta fase, se comprueban cuáles son los ganadores de la subasta, y se publican los resultados. Para esto, ya se tienen las ofertas
      válidas, determinadas por la fase de revelación, estas se ordenan crecientemente por el porciento de interés que ofrecen, es decir
      las que menor porciento de interés tiene el préstamo van primero, dado que son las más convenientes para el deudor. En caso de tener
      el mismo porciento se desempata por la oferta con menor id. 

      Luego de ordenadas las ofertas, se comienzan a aceptar ofertas hasta lograr la cantidad total que se necesita para cubrir el valor
      de los bonos ofertados. Para esto, se comienza con la oferta con menor porciento de interés, 
      y se acepta la oferta, es decir, disminuye la cantidad de bonos ofertados. Luego se pasa a la siguiente oferta, y se acepta la oferta, 
      y así sucesivamente hasta que se agoten los bonos ofertados o hasta agotar todas las ofertas válidas.

      Se paga un único porciento de interés a todos los ganadores de la subasta, el cual es el porciento de interés de la última oferta
      aceptada, es decir, la oferta aceptada con mayor porciento de interés.

      En caso de que para un porciento de interés se tengan varias ofertas con ese mismo porciento, se acepta primero la oferta con menor 
      id. Por consiguiente, si en el último porciento de interés aceptado se tienen varias ofertas, el desempate está dado por el tiempo
      de revelación de la oferta (que es lo que determina el id de la oferta). Esto estimula a los postores a revelar sus ofertas lo más
      pronto posible, para tener una mayor probabilidad de estar entre los ganadores la subasta.

      En caso de que la última oferta aceptada sobrepase la cantidad de bonos ofertados, se acepta la oferta parcialmente, es decir, se
      acepta solamente la cantidad que se necesita para cubrir el valor de los bonos ofertados.

      Luego de que ya se tienen las ofertas ganadoras, el dinero bloqueado de las ofertas restantes (si quedara alguna) es puesto a 
      disposición de sus respectivos postores, para que puedan retirarlo.

      \subsection{Finalización}
      Esta fase está estrechamente ligada con la anterior, y se hacen una a continuación de la otra. Pero para mejor entendimiento se 
      puso en una fase aparte. En esta fase, se publican los resultados de la subasta, es decir, se publican las ofertas ganadoras
      (dirección, porciento y cantidad a prestar de cada una). Además, se publica el porciento de interés que se le pagará a los ganadores.
      Y seguidamente se transfiere el dinero de los postores ganadores, en este caso convertidos en prestamistas, a la dirección del 
      beneficiario de la subasta, que sería el deudor de los préstamos.

      Por último el contrato inteligente activa una bandera que indica que la subasta ya ha finalizado, y que no se pueden hacer más
      ofertas, ni revelaciones, ni nada relacionado con la subasta, solamente retiros del dinero de los postores que tengan dinero disponible.

      El contrato fue diseñado para subasta de bonos soberanos, pero puede ser empleado para cualquier otra subasta del mercado de deuda.
      Y también puede ser fácilmente adaptado para realizar otro tipo de subastas.

  \section{Seguridad}
    \subsection{Obligación de pago del beneficiario}
      En la implementación realizada, se asume la fiabilidad del beneficiario de la subasta, a la postre deudor. No se llega a implementar
      ninguna medida para asegurar el pago a los prestamistas, dado que depende mucho de las condiciones de la subasta y se escapa un poco
      del alcance que pudiera tener el contrato inteligente, sin embargo, se 
      propondrán algunas propuestas, que pudieran ser empleadas para garantizar o al menos mejorar la confianza de los postores hacia
      el beneficiario.

      \begin{enumerate}
        \item El beneficiario tenga que depositar algún activo como colateral en el contrato inteligente, que sea igual al valor total o al menos
        a los intereses a pagar por los bonos que
        se ofertan. De esta manera, si el beneficiario no paga, el contrato inteligente puede pagar a los prestamistas con el dinero
        depositado por el beneficiario.
        \item Liberación de fondos en partes. En lugar de liberar el dinero de los prestamistas al final de la subasta, se puede liberar el 
        dinero de los prestamistas en partes, es decir, liberar el dinero de los prestamistas en partes, proporcionalmente a los intereses 
        que se
        van pagando. De esta manera, si el beneficiario no paga, el contrato inteligente puede pagar a los prestamistas con el dinero
        congelado en el contrato inteligente.
        \item Tokenizar los bonos. En lugar de ofertar los bonos, se puede ofertar tokens que representen los bonos. De esta manera, estos
        tokens servirían como garantía de que el beneficiario pagará los intereses de los bonos. Quizás hasta se podrían legalizar estos
        tokens, de manera de que se pudieran incurrir en acciones legales en caso de que el beneficiario no pague los intereses de los
        bonos o la devolución total  del préstamo luego del vencimiento de este. Estos tokens pudieran también servir para que los prestamistas
        puedan vender sus tokens en el mercado secundario, y así recuperar parte de su dinero invertido. Creando así un mercado secundario
        para los bonos, que es algo que no existe en el mercado de deuda cubano actualmente y que fomentaría la liquidez de los bonos y el
        interés de los inversores en estos.
      \end{enumerate}

      El autor opina que cada una de estas opciones puede ser viable en algún caso particular, sin embargo, piensa que la mejor y más 
      conveniente opción 
      es la tercera, ya que es la que mejor garantiza el pago de los intereses y la devolución del préstamo, y además, tiene múltiples
      beneficios adicionales, para los prestamistas y el mercado de deuda cubano en general.

    \subsection{Keccak256}
      Keccak está basada en una novedosa propuesta llamada construcción de esponja. La construcción de esponja está basada en una amplia
      función aleatoria o permutación aleatoria, y permite la entrada (“absorber” metafóricamente en terminología de esponja) de cualquier 
      cantidad de datos,
      y la salida (“exprimir”) cualquier cantidad de datos, mientras actúa como una función pseudoaleatoria con respecto a todas las entradas
      anteriores. Esto provoca una gran flexibilidad. \parencite{bertoni2007}

      El algoritmo Keccak es el trabajo de Guido Bertoni, Joan Daemen, Michael Peeters y Gilles Van Assche. Está basado en los diseños de 
      hash PANAMA y RadioGatún. En el año 2006 el \textit{National Institute of Standards and Technology} (NIST) organizó una nueva edición de la competición para la creación de una nueva función 
      para los estándares de Secure Hash Algorithm (SHA), el SHA-3. Al no existir un ataque significativo demostrado en SHA-2, el nuevo 
      estándar SHA-3 no lo reemplazará. La NIST ha mencionado que debido a exitosos ataques a los estándares MD5, SHA-0 y SHA-1, es 
      necesario una alternativa llamada SHA-3. \parencite{stevens2017}

      Las admisiones de proyectos comenzaron en el año 2008. Keccak fue aceptado como uno de los 51 candidatos. En julio de 2009, 14 
      algoritmos pasaron a la segunda ronda y Keccak avanzó a la ronda final en diciembre de 2010. Durante el periodo de la competición, se 
      les permitió a los participantes corregir problemas descubiertos en sus algoritmos. Keccak hizo algunos cambios como el número de 
      rondas, se amplió de ${\displaystyle 12-\ell}$ a ${\displaystyle 12-2\ell }$. En 
      octubre de 2012 fue seleccionado como el ganador de la competición. En el año 2014 la NIST publicó la documentación técnica del 
      algoritmo y fue aprobado en agosto de 2015, para así convertirse en el nuevo estándar SHA-3. \parencite{nist2015}

      En el caso de Bitcoin, el uso del algoritmo de hash SHA256 está bastante extendido,
      además es utilizado en multitud de implementación de funcionalidades de la cadena
      de bloques de Bitcoin. En cambio, Ethereum hace uso del algoritmo de hash Keccak-
      256, aprobado por la NIST en agosto del 2015, convirtiéndose en el nuevo estándar
      SHA-3.
      Cabe destacar que debido a que la aprobación por parte del NIST de este algoritmo fue más tardía que el desarrollo de Ethereum, el 
      estándar final de SHA-3 adoptado por
      la NIST hace referencia al estándar “FIPS-202 SHA-3”, el cual sufrió ligeros cambios
      en sus parámetros a diferencia del algoritmo Keccak-256 implementado en Ethereum.

      \begin{verbatim}
        Keccak256("")
        = c5d2460186f7233c927e7db2dcc703c0e500b653ca82273b7bfad8045d85a470

        SHA3("")
        = a7ffc6f8bf1ed76651c14756a061d662f580ff4de43b49fa82d80a4b80f8434a
      \end{verbatim}

      Como se produjeron ligeros cambios, las salidas de estos algoritmos no son iguales
      entre sí en el caso de que haya entradas iguales. Para comprobarlo, como se hace en
      el ejemplo anterior, se puede observar los valores en ambos casos de uno entrada vacía, lo
      que da como resultado dos salidas distintas. Por lo tanto, a pesar de que en múltiples
      papers y documentos de Ethereum se haga referencia a SHA-3, realmente se quiere
      hacer referencia al uso de Keccak-256. \parencite{taibo2022}

      % other post: https://www.geeksforgeeks.org/difference-between-sha-256-and-keccak-256/
