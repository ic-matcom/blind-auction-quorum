\chapter*{Introducción}\label{chapter:introduction}
\addcontentsline{toc}{chapter}{Introducción}

  \hspace*{}

  Cuando el 31 de octubre de 2008 Satoshi Nakamoto publicó el artículo \textit{Bitcoin: A Peer-to-Peer Electronic Cash System} (documento 
  técnico original de la bien conocida y primera criptomoneda: Bitcoin) ~\parencite{satoshi2008}, creó las bases de una tecnología que 
  está revolucionando al mundo, y el autor no se refiere al que bien pudiera ser el sistema de pago que sustituya al dólar y al dinero 
  fiat en un futuro, sino a la blockchain.
  
  Blockchain se traduce como cadena de bloques. Básicamente, blockchain es un conjunto de tecnologías que permiten llevar un registro 
  seguro, descentralizado, sincronizado y distribuido de operaciones digitales, sin necesidad de la intermediación de terceros \parencite{solunion2021}.

  En ese sentido, la definición más completa es la dada por Don \& Alex Tapscott en su libro Blockchain Revolution 
  : “un libro de 
  contabilidad digital incorruptible de transacciones económicas que se puede programar para registrar no solo transacciones financieras, 
  sino prácticamente todo lo que tiene valor”~\parencite{tapscott2016blockchain}. Cada uno de los bloques de datos se encuentra protegido y vinculado entre sí 
  criptográficamente. Las transacciones no las verifica un tercero, sino la red 
  de nodos (computadores conectados a la red), que también es la que autoriza en consenso cualquier actualización en la blockchain \parencite{solunion2021}.

  A finales de 2013, Vitalik Buterin publica el que luego se convertiría en el documento técnico (white paper) de Ethereum 
  \parencite{buterin2013}. Este joven, quien hasta ese momento era uno de los
  programadores involucrado en el ecosistema Bitcoin, había notado el potencial de la criptografía para el desarrollo de aplicaciones 
  descentralizadas. No obstante, su propuesta de crear un lenguaje de scripting para Bitcoin, que hiciera esto posible, no tuvo resonancia 
  suficiente. Fue entonces cuando se propuso el desarrollo de una red independiente, con su propia infraestructura, para el desarrollo de 
  un criptoactivo y una cadena de bloques capaz de soportar aplicaciones descentralizadas. Y el 30 de julio de 2015, Vitalik conjuntamente
  con otros programadores pusieron en línea la blockchain de Ethereum. \parencite{diaz2018}
  
  La red de Ethereum llevó a la práctica un nuevo concepto, los contratos inteligentes, en inglés conocidos como \textit{smart contracts}. 
  La definición más simple al respecto es que se trata de contratos que tienen la capacidad de cumplirse de forma automática una vez que 
  las partes han acordado los términos. Su nombre hace recordar a los contratos legales firmados en papel. Pero a pesar de que tienen 
  cosas en común, son totalmente diferentes. 

  Los contratos inteligentes son programas informáticos. No están escritos en lenguaje natural, sino en código virtual. Son un 
  tipo de software que se programa, como cualquier otro software, para llevar a cabo una tarea o serie de tareas determinadas de acuerdo a 
  las instrucciones previamente introducidas. Su cumplimiento, por tanto, no está sujeto a la interpretación de ninguna de las partes: si 
  el evento A sucede, entonces la consecuencia B se pondrá en marcha de forma automática. Su implicación legal ha caído -como toda la 
  tecnología relacionada a Bitcoin- en una zona gris. No se requiere de ningún intermediario de confianza (como una notaría), pues este 
  papel lo adopta el código informático, que asegurará sin dudas el cumplimiento de las condiciones. Por tanto, se reducen tiempo y costes 
  significativamente. ~\parencite{smartcontract}

  % A smart contract is a computer program or a transaction protocol that is intended to automatically execute, control or document legally 
  % relevant events and actions according to the terms of a contract or an agreement.[1][2][3][4] The objectives of smart contracts are the 
  % reduction of need for trusted intermediators, arbitrations costs, fraud losses, as well as the reduction of malicious and accidental 
  % exceptions.[5][2]

  Las ventajas son obvias, y pueden reducirse a tres palabras: autonomía, seguridad y confianza. Utilizando contratos inteligentes ya no 
  resulta necesario recurrir a un tercero —como un abogado o un notario—, que además de que pueden provocar errores, ocasiona gastos 
  significativos. La blockchain es capaz de resguardar la información en una red cifrada que puede consultarse desde cualquier lugar del 
  mundo, por lo que la velocidad y seguridad saltan a la vista.

  Con esta nueva tecnología se puede crear una gran cantidad de nuevas aplicaciones para hacer trámites y transacciones hasta ahora
  difíciles de realizar con las tecnologías existentes, o simplemente mejorar servicios gracias a la descentralización
  de la blockchain y de los contratos inteligentes.

  Una de estas aplicaciones de los contratos inteligentes está dada a las subastas. Una subasta es una venta generalmente pública en la 
  que se adjudica una cosa, especialmente bienes o cosas de valor, a la persona que ofrece más dinero por ella. La blockchain puede y 
  está cambiando, la manera en la que se hacen las subastas, ya sin necesidad de un subastador o de alguna entidad que haga de mediador. 
  Ya muchas casas de apuestas han actualizado sus políticas, para adaptarse a los nuevos métodos, de hacer subastas. 
  % https://tezro.com/
  % https://bitify.com/
  % https://nft.christies.com/  Founded in 1766, Christie’s is a world-leading art and luxury business.
  % https://portion.io/     Portion is the premier online marketplace connecting artists and collectors through Blockchain technology to 
  % easily sell, invest and own art and collectibles with complete transparency.
  % https://www.coindesk.com/markets/2021/07/09/sothebys-sells-rare-diamond-for-123m-in-crypto/

  Los mercados de deuda pública o de bonos soberanos(o del estado) siempre han hecho uso de subastas para hacer sus ventas. Con la
  blockchain se abre una nueva puerta para una forma segura, eficiente y sencilla de efectuar estas subastas. 

  Específicamente, en el presente trabajo se estudian las subastas a ciegas. Una subasta a ciegas es aquella en la que solo el ofertante 
  sabe el monto de su oferta y nadie más. Él no conoce las ofertas de los demás y viceversa.

  La implementación de subastas a ciegas sobre blockchain presupone una dificultad, pues toda información que se almacena es pública y 
  verificable por cualquiera que esté conectado a la red de nodos. La solución a esto podría ser el cifrado de las ofertas que hacen 
  los pujadores. Por tanto, el problema consiste en la selección del algoritmo adecuado para el desarrollo de un sistema de 
  subasta a ciegas sobre Quorum.

  El \textbf{objetivo} general de este trabajo es el diseño e implementación de un conjunto de algoritmos que permita el desarrollo de 
  subastas a ciegas sobre Quorum. Quorum es una blockchain basada en Ethereum.

  Para lograr el objetivo general se definen los siguientes objetivos específicos:

  \begin{enumerate}
    \item Identificar las soluciones técnicas y tecnologías que se emplean para el desarrollo de aplicaciones relacionadas 
    con subasta electrónica.

    \item Valorar las posibilidades de Quorum como plataforma para el desarrollo de aplicaciones de este tipo.

    \item Estudio de Solidity como lenguaje de programación para el desarrollo de contratos inteligentes sobre Quorum.

    \item Implementar contratos inteligentes (algoritmos) que permitan efectuar subastas a ciegas.

  \end{enumerate}

  La memoria escrita está organizada en 3 Capítulos.
  En el Capítulo 1 se aborda el tema de las subastas, qué beneficios presenta la blockchain para el desarrollo de subastas y 
  una comparación entre algunos tipos de protocolos de subastas a ciegas sobre blockchain. En el Capítulo 2 se explica la 
  utilización de la blockchain de Quorum para el desarrollo de un sistema de subastas a ciegas. El Capítulo 3 está dedicado a la 
  evaluación de los resultados y mostrar el desempeño del método propuesto. Finalmente, se dan las conclusiones de la investigación, 
  recomendaciones, así como la bibliografía y los anexos necesarios para la mejor comprensión de la propuesta.
