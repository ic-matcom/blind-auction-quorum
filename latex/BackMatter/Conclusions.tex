\begin{conclusions}
    La blockchain y los contratos inteligentes surgen, luego del 2015 y la aparición de la blockchain de Ethereum, como una herramienta eficaz para hacer frente a problemas de seguridad y centralización, permitió disponer de la capacidad de crear transacciones y programas incorruptibles, que se mantienen públicos a la vista de todos, haciendo que algunos procesos antes oscuros y cerrados, sean ahora totalmente transparentes y abiertos a cualquier persona.

    Gracias a esta tecnología, surge la idea de crear subastas en la blockchain, aportándole así seguridad
    y fiabilidad al proceso. Pero con esta total transparencia de la blockchain surgen algunas dificultades
    para realizar algunos tipos de subasta. ¿Cómo lograr ejecutar una subasta a ciega, si toda la información
    es pública?

    En la literatura consultada se presentan varias propuestas, casi todas relacionadas directamente
    con la criptografía para ocultar las ofertas de la subasta. En la propuesta implementada se utiliza con
    este objetivo la función \textit{hash keccak256}, la cual es altamente probada y utilizada para 
    proteger datos (en este caso ofertas de la subasta) de los demás usuarios de la subasta, incluyendo 
    el subastador.

    Para el desarrollo del contrato inteligente se utilizó el lenguaje Solidity, el cual es un lenguaje
    especializado en la creación de contratos inteligentes específicamente creado para la red de 
    Ethereum, pero que redes posteriormente creadas también han adoptado su uso.

    A pesar de no haber llegado a desplegarse el contrato en la red de Quorum, el autor considera que las
    características de esta red son ideales para desarrollar un contrato de este tipo. Como característica
    a destacar es que las transacciones realizadas en esta red, al ser una red privada y no pública (como 
    Ethereum) son libres de costo, hecho que facilita la realización satisfactoria
    de todos los procesos y transacciones que requiere el contrato inteligente propuesto.

    Además, el contrato inteligente implementado, según las pruebas realizadas, necesita que la cantidad
    de ofertas sean limitadas, para un correcto funcionamiento del algoritmo y el entorno de una red
    privada como Quorum permite tener más control sobre eso, sin afectar la seguridad y cumplimiento
    total de los procesos de la subasta en el contrato.
\end{conclusions}
