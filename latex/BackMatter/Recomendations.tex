\begin{recomendations}
    Luego de la investigación, la implementación y las pruebas realizadas en el presente trabajo. Se 
    proponen algunas recomendaciones para analizar e investigar en futuros trabajos.

    Se hace necesario un estudio profundo de los tipos de almacenamiento que admite Solidity y los tipos de
    datos usados para implementar el algoritmo, para optimizar el uso de la memoria empleada por el algoritmo
    propuesto, recurso muy costoso en las redes blockchain.

    Se recomienda además la implementación de un \tectit{Heap} (Monticulo) en detrimento del Quick Sort,
    para escoger las ofertas ganadoras de la subasta y hacer una comparación, para ver con cuál de los
    dos algoritmos la fase de verificación de los ganadores utiliza menos gas, y por tanto, menos recursos.

    Dado que las pruebas en la red blockchain local fueron satisfactorias, se recomienda el despliegue del
    contrato implementado en la red de Quorum.

    Por último, se hace necesario la implementación de un mecanismo para asegurar el cumplimiento del
    beneficiario (ofertante de los bonos) de su parte en el acuerdo. La opción recomendada y más 
    factible según el autor, es tokenizar los bonos, y darlos a los postores de las ofertas ganadoras,
    en proporción a la cantidad de bonos comprados por este.

\end{recomendations}
